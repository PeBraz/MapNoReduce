\section {Implementation}

The computation in our implementation takes the input key/value pairs from input files, where the keys are the numbers of the line of the file being read and the values are the content

\subsection {Execution Overview}


{\LARGE{Isto é preciso modificar!!!}}

 \it\tiny
The Map invocations are distributed across multiple
machines by automatically partitioning the input data
To appear in OSDI 2004 3
into a set of M splits. The input splits can be processed
in parallel by different machines. Reduce invocations
are distributed by partitioning the intermediate key
space into R pieces using a partitioning function (e.g.,
hash(key) mod R). The number of partitions (R) and
the partitioning function are specified by the user.
Figure 1 shows the overall flow of a MapReduce operation
in our implementation. When the user program
calls the MapReduce function, the following sequence
of actions occurs (the numbered labels in Figure 1 correspond
to the numbers in the list below):

\begin{enumerate}
\item The MapReduce library in the user program first
splits the input files into M pieces of typically 16
megabytes to 64 megabytes (MB) per piece (controllable
by the user via an optional parameter). It
then starts up many copies of the program on a cluster
of machines.
\item{One of the copies of the program is special – the
master. The rest are workers that are assigned work
by the master. There are M map tasks and R reduce
tasks to assign. The master picks idle workers and
assigns each one a map task or a reduce task.}
\item{A worker who is assigned a map task reads the
contents of the corresponding input split. It parses
key/value pairs out of the input data and passes each
pair to the user-defined Map function. The intermediate
key/value pairs produced by the Map function
are buffered in memory.}
\item{Periodically, the buffered pairs are written to local
disk, partitioned into R regions by the partitioning
function. The locations of these buffered pairs on
the local disk are passed back to the master, who
is responsible for forwarding these locations to the
reduce workers.}
\item{When a reduce worker is notified by the master
about these locations, it uses remote procedure calls
to read the buffered data from the local disks of the
map workers. When a reduce worker has read all intermediate
data, it sorts it by the intermediate keys
so that all occurrences of the same key are grouped
together. The sorting is needed because typically
many different keys map to the same reduce task. If
the amount of intermediate data is too large to fit in
memory, an external sort is used.}
\item{The reduce worker iterates over the sorted intermediate
data and for each unique intermediate key encountered,
it passes the key and the corresponding
set of intermediate values to the user's Reduce function.
The output of the Reduce function is appended
to a final output file for this reduce partition.
7. When all map tasks and reduce tasks have been
completed, the master wakes up the user program.
At this point, the MapReduce call in the user program
returns back to the user code.}
\item{After successful completion, the output of the mapreduce
execution is available in the R output files (one per
reduce task, with file names as specified by the user).
Typically, users do not need to combine these R output
files into one file – they often pass these files as input to
another MapReduce call, or use them from another distributed
application that is able to deal with input that is
partitioned into multiple files.}
\end{enumerate}


\subsection {Master Data Structures}

The master keeps several data structures. For each map
task and reduce task, it stores the state (idle, in-progress,
or completed), and the identity of the worker machine
(for non-idle tasks).
The master is the conduit through which the location
of intermediate file regions is propagated from map tasks
to reduce tasks. Therefore, for each completed map task,
the master stores the locations and sizes of the R intermediate
file regions produced by the map task. Updates
to this location and size information are received as map
tasks are completed. The information is pushed incrementally
to workers that have in-progress reduce tasks.


