\section{Introduction}

MapReduce is a programming model and an associated implementation for processing and generating large data sets with a parallel, distributed algorithm on a cluster being first introduced by Google in 2004. 
By using this programming paradigm it is possible to reach massive scalability across hundreds or thousands of servers.



The term MapReduce actually refers to two separate and distinct tasks. 
\begin{itemize}
\item \textbf{map :} takes a set of data and converts it into another set of data, where individual elements are broken down into tuples (key/value pairs).
\item \textbf{reduce :} takes the output from a map as input and combines those data tuples into a smaller set of tuples. As the sequence of the name MapReduce implies, the reduce job is always performed after the map job.
\end{itemize}


PADIMapNoReduce is a simplified implementation of the MapReduce middleware and programming model, concerning only about the mapping part.

The Map invocations are distributed across multiple machines by automatically partitioning the input data into a set of splits of size S. The input splits can be processed in parallel by different machines(workers), ensuring that for each job submitted, all the input data is processed.

We present an implementation which makes workers take the functionalities of the job tracker, meaning the worker may behave differently depending on the situation.
